\chapter*{Einleitung}

\addcontentsline{toc}{chapter}{\protect\numberline{}Einleitung}%

\lettrine{F}{raktale} sind faszinierende Strukturen, die uns allgegenwärtig umgeben.
Von der verzweigten Struktur eines Baumes bis hin zu den Wolken am Himmel, die sich in immer kleiner werdende Strukturen aufteilen, sind Fraktale ein fester Bestandteil unserer natürlichen Umgebung.
Aber auch in der Mathematik sind Fraktale zu finden. \newline
Der Begriff \textit{Fraktal} wurde um 1975 vom Mathematiker Benoît Mandelbrot geprägt und beschreibt natürliche und künstliche Formen, die bestimmte geometrische Eigenschaften aufweisen. \newline
Fraktale zeichnen sich durch ihre \textit{Selbstähnlichkeit} auf unterschiedlichen Größenskalen aus. Das bedeutet, dass Teile der Struktur in verschiedenen Vergrößerungen ähnliche Formen aufweisen. \newline
\hfill \break
Im theoretischen Teil dieser komplexen Leistung werde ich mich mit der mathematischen Theorie hinter Fraktalen beschäftigen, insbesondere mit der \textit{Mandelbrot-Menge} und den \textit{Julia-Mengen}. \newline
Der praktische Teil beinhaltet die Algorithmen sowie deren Implementierung zur Darstellung von Fraktalen mithilfe von Computergrafik und der Programmiersprache \textit{C++}, wobei insbesondere die \textit{OpenGL}-Bibliothek genutzt wird. \newline
\hfill \break
Ziel dieser komplexen Leistung ist es, nicht nur die mathematischen Grundlagen von Fraktalen zu verstehen, sondern auch die Ästhetik von Fraktalen zu verdeutlichen, sowie auf die Anwendung von Fraktalen in der realen Welt einzugehen.