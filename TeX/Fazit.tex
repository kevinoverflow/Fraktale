\chapter{Fazit}
\thispagestyle{fancy} % Manually set the page style

Die Auseinandersetzung mit der faszinierenden Welt der Mathematik durch die
Visualisierung von Fraktalen hat nicht nur Einblicke in mathematische Konzepte,
sondern auch in die Computergrafik gegeben aber auch die ästhetische Schönheit
dieser abstrakten Strukturen hervorgehoben. \hfill \break \newline \noindent
Der technologische Fortschritt, insbesondere im Bereich der Computergrafik und
Programmierung, hat es ermöglicht, Fraktale in einer Weise zu visualisieren,
die vor einigen Jahrzehnten undenkbar war. Das Programm \textit{Fraktale}
bietet die Möglichkeit, interaktiv in die Welt der Fraktale einzutauchen, sie
zu erforschen und ihre Schönheit zu erleben.

\hfill \break  \noindent
Die Anwendungsmöglichkeiten von Fraktalen erstrecken sich über viele verschiedene Bereiche, von reiner Kunst bis hin zu wissenschaftlichen Analysen. Die Selbstähnlichkeit und Komplexität von Fraktalen finden in Bildkompression, digitaler Kunst, Signalverarbeitung, Wettermodellierung und Finanzanalyse Anwendung.

\hfill \break  \noindent
Insgesamt zeigt diese komplexe Leistung, wie moderne Technologien und mathematische Konzepte zusammenarbeiten können, um eine Anwendung zu entwickeln, die sowohl informativ als auch ästhetisch ansprechend ist.