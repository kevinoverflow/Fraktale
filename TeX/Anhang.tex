\titlespacing*{\chapter}{0pt}{-30pt}{20pt}

\chapter{Anhang}
\thispagestyle{fancy} % Manually set the page style

\section{Quellcode}
Der Quellcode und das Programm sind über den beigelegten USB-Stick oder über
folgenden GitHub Link verfügbar: \newline
\url{https://www.github.com/kevinoverflow/Fraktale}
% \section{Abbildungen} %
\section{Bibliotheken}
\subsubsection*{CMake - \url{https://cmake.org/}}
\textit{Ein plattformübergreifendes Open-Source-Build-System.}
\subsubsection*{CMakeRC - \url{https://github.com/vector-of-bool/cmrc}}
\textit{Ein Ressourcencompiler in einem einzigen CMake-Skript.}
\subsubsection*{Dear ImGui - \url{https://github.com/ocornut/imgui}}
\textit{Eine einfache Immediate-Mode-Grafikbenutzeroberfläche für C++.}
\subsubsection*{GLFW - \url{https://www.glfw.org/}}
\textit{Eine plattformübergreifende Bibliothek zur Erstellung von Fenstern mit OpenGL-Kontexten.}
\subsubsection*{GLEW - \url{http://glew.sourceforge.net/}}
\textit{Die OpenGL Extension Wrangler Library für eine einfachere Handhabung von OpenGL-Erweiterungen.}
\subsubsection*{GLM - \url{https://glm.g-truc.net/0.9.9/index.html}}
\textit{OpenGL Mathematics (GLM) ist eine rein Header basierte C++-Mathematikbibliothek für Grafiksoftware, basierend auf den Spezifikationen der OpenGL Shading Language (GLSL).}
\newpage
\section{Literaturverzeichnis}
\printbibliography[heading=none]