\titlespacing*{\chapter}{0pt}{-30pt}{20pt}

\chapter{Anhang}
\thispagestyle{fancy} % Manually set the page style

\section{Quellcode}
Der Quellcode und das Programm sind über den beigelegten USB-Stick oder über
folgenden GitHub Link verfügbar: \newline
\url{https://www.github.com/kevinoverflow/Fraktale}
% \section{Abbildungen} %
\section{Bibliotheken}
\subsubsection*{GLFW}
\textit{Eine plattformübergreifende Bibliothek zur Erstellung von Fenstern mit OpenGL-Kontexten.}
\subsubsection*{GLEW}
\textit{Die OpenGL Extension Wrangler Library für eine einfachere Handhabung von OpenGL-Erweiterungen.}
\subsubsection*{GLM}
\textit{OpenGL Mathematics (GLM) ist eine rein Header basierte C++-Mathematikbibliothek für Grafiksoftware, basierend auf den Spezifikationen der OpenGL Shading Language (GLSL).}
\subsubsection*{Dear ImGui}
\textit{Eine einfache Immediate-Mode-Grafikbenutzeroberfläche für C++.}
\subsubsection*{CMake}
\textit{Ein plattformübergreifendes Open-Source-Build-System.}
\subsubsection*{CMakeRC}
\textit{Ein Ressourcencompiler in einem einzigen CMake-Skript.}

\section{Literaturverzeichnis}
Literaturverzeichnis